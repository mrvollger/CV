%%%%%%%%%%%%%%%%%%%%%%%%%%%%%%%%%%%%%%%%%
% "ModernCV" CV and Cover Letter
% LaTeX Template
% Version 1.11 (19/6/14)
%
% This template has been downloaded from:
% http://www.LaTeXTemplates.com
%
% Original author:
% Xavier Danaux (xdanaux@gmail.com)
%
% License:
% CC BY-NC-SA 3.0 (http://creativecommons.org/licenses/by-nc-sa/3.0/)
%
% Important note:
% This template requires the moderncv.cls and .sty files to be in the same 
% directory as this .tex file. These files provide the resume style and themes 
% used for structuring the document.
%
%%%%%%%%%%%%%%%%%%%%%%%%%%%%%%%%%%%%%%%%%

%----------------------------------------------------------------------------------------
%	PACKAGES AND OTHER DOCUMENT CONFIGURATIONS
%----------------------------------------------------------------------------------------

\documentclass[11pt,a4paper,sans]{moderncv} % Font sizes: 10, 11, or 12; paper sizes: a4paper, letterpaper, a5paper, legalpaper, executivepaper or landscape; font families: sans or roman

\moderncvstyle{casual} % CV theme - options include: 'casual' (default), 'classic', 'oldstyle' and 'banking'
\moderncvcolor{black} % CV color - options include: 'blue' (default), 'orange', 'green', 'red', 'purple', 'grey' and 'black'

\usepackage{lipsum} % Used for inserting dummy 'Lorem ipsum' text into the template
\usepackage[scale=0.75]{geometry} % Reduce document margins


\usepackage[T1]{fontenc}
\usepackage{lmodern}
%\setlength{\hintscolumnwidth}{3cm} % Uncomment to change the width of the dates column
%\setlength{\makecvtitlenamewidth}{10cm} % For the 'classic' style, uncomment to adjust the width of the space allocated to your name

%----------------------------------------------------------------------------------------
%	NAME AND CONTACT INFORMATION SECTION
%----------------------------------------------------------------------------------------
\newcommand\Colorhref[3][cyan]{\href{#2}{\small\color{#1}#3}}

\firstname{Mitchell Robert} % Your first name
\familyname{Vollger} % Your last name

% All information in this block is optional, comment out any lines you don't need
\title{Curriculum Vitae}
%\address{123 Broadway}{City, State 12345}
%\mobile{(000) 111 1111}
%\phone{(000) 111 1112}
%\fax{(000) 111 1113}
%\email{john@smith.com}
%\homepage{staff.org.edu/~jsmith}{staff.org.edu/$\sim$jsmith} % The first argument is the url for the clickable link, the second argument %is the url displayed in the template - this allows special characters to be displayed such as the tilde in this example

%\extrainfo{additional information}

%\photo[70pt][0.4pt]{pictures/picture} % The first bracket is the picture height, the second is the thickness of the frame around the picture (0pt for no frame)

%\quote{"A witty and playful quotation" - John Smith}

%----------------------------------------------------------------------------------------

\begin{document}
\makecvtitle % Print the CV title

\section{\textbf{Personal Data}}
\hspace{25 mm}\cvitem{Legal name}{\hspace{4 mm} Mitchell Robert Vollger}
\hspace{25 mm}\cvitem{Birth}{\hspace{4 mm} 3 November 1992, Carson City (Nevada)}

\subsection{\textbf{Email}}
\hspace{25 mm}\cvitem{Primary}{\hspace{4 mm}mvollger@uw.com}
\hspace{25 mm}\cvitem{Secondary}{\hspace{4 mm}mrvollger@gmail.com}

\subsection{\textbf{Phone}}
\hspace{25 mm}\cvitem{Mobile}{\hspace{4 mm}(707) 407-8732}
\subsection{\textbf{Address}}
\hspace{25 mm} \cvitem{Primary}{\hspace{4 mm}1540 Eastlake Ave. East Apt 404, Seattle Washington, 98102}

%----------------------------------------------------------------------------------------
%	EDUCATION SECTION
%----------------------------------------------------------------------------------------

\section{\textbf{Education}}

%\texttt{\href{http://www.google.com}}

\cventry{Sep 2016 -- March 2021}{Ph.D in Genome Sciences}{University of Washington}{Seattle, Washington}{GPA -- 3.86}{}  % Arguments not required can be left empty


\cventry{Sep. 2011 -- June 2015}{B.S.E. in Computer Science Engineering}{Princeton University}{Princeton, New Jersey}{\textit{GPA -- 3.13}. Student of the \textcolor{blue}{\small{\href{http://www.princeton.edu/integratedscience/}{Integrated Science Curriculum}}}}{}  % Arguments not required can be left empty
 \cventry{Sep. 2011 -- June 2015 }{Certificate in \textcolor{blue}{\href{http://www.princeton.edu/integratedscience/certificate/}{Quantitative and Computational Biology}}}{Princeton University}{Princeton, New Jersey}{\textit{GPA -- 3.48}}{}  % Arguments not required can be left empty
\cventry{Sep. 2008 -- June 2011 }{Associate of Arts Degree in Mathematics}{College of the Redwoods}{Eureka, California}{\textit{GPA -- 4.00}}{}
\cventry{Sep. 2008-- June 2011}{Associate of Arts Degree in Science}{College of the Redwoods}{Eureka, California}{\textit{GPA -- 4.00}}{}


%\cventry{2007--2010}{Bachelor of Business Studies}{The University of California}{Berkeley}{\textit{GPA -- 7.5}}{Specialized in Commerce}
\section{\textbf{Postgraduate Training}}
\cventry{March 2021 -- Present }{Postdoctoral Fellow at Genome Sciences}{University of Washington}{Seattle, Washington}{}{}  % Arguments not required can be left empty

%\section{\textbf{Faculty Positions Held}}
%None
%\section{\textbf{Hospital Positions Held}}
%one

%----------------------------------------------------------------------------------------
%	honors SECTION
%----------------------------------------------------------------------------------------

\section{\textbf{Honors}}
\cvitem{2011}{Graduated with Highest Honors, College of the Redwoods}
\cvitem{2011}{National Hispanic Recognition Program Scholar}
\cvitem{2011}{National Merit Scholarship Semifinalist}
\cvitem{2011}{Valedictorian, Academy of the Redwoods}


%----------------------------------------------------------------------------------------
%	Random SECTION
%----------------------------------------------------------------------------------------


%\section{\textbf{Board Certifications}}
%None

%\section{\textbf{Current Licenses to Practice}}
%None

\section{\textbf{Professional Organizations}}
\cvitem{2021}{Member of American Society of Human Genetics (ASHG)}

%\section{\textbf{Teaching Responsibilities}}
%None

%\section{\textbf{Special National Responsibilities}}
%None

%\section{\textbf{Special Local Responsibilities}}
%None

%\section{\textbf{Editorial Responsibilities}}
%None


\section{\textbf{Research Funding}}
\cvitem{Fall 2017 - Fall 2019}{BDGN, Big Data in Genomics and Neuroscience. Awarded for two years.}
\cvitem{Fall 2016 - Fall 2017}{NIH/NHGRI T32, through the Genome Training Grant. Awarded for two years.}


%----------------------------------------------------------------------------------------
%	Publications SECTION
%----------------------------------------------------------------------------------------

\section{\textbf{Bibliography}}
\subsection{\textbf{Primary author}}

\cvitem{\textbf{---}}{\textbf{Vollger MR}, Logsdon GA, Audano PA, Sulovari A, Porubsky D, Peluso P, et al. Improved assembly and variant detection of a haploid human genome using single-molecule, high-fidelity long reads. Ann Hum Genet. 2020;84: 125–140. doi:10.1111/ahg.12364}

\cvitem{\textbf{---}}{  \textbf{Vollger MR}, Dishuck PC, Sorensen M, Welch AE, Dang V, Dougherty ML, Graves-Lindsay TA, Wilson RK, Chaisson MJP, Eichler EE. 2019. Long-read sequence and assembly of segmental duplications. Nat Methods. doi:10.1038/s41592-018-0236-3 }

\subsection{\textbf{Collaborative author}}

\cvitem{\textbf{---}}{  Miga KH, Koren S, Rhie A, \textbf{Vollger MR}, Gershman A, Bzikadze A, et al. Telomere-to-telomere assembly of a complete human X chromosome. Nature. 2020;585: 79–84. doi:10.1038/s41586-020-2547-7 }

\cvitem{\textbf{---}}{ Logsdon GA, \textbf{Vollger MR}, Eichler EE. Long-read human genome sequencing and its applications. Nat Rev Genet. 2020. doi:10.1038/s41576-020-0236-x }

\cvitem{\textbf{---}}{Nurk S, Walenz BP, Rhie A, \textbf{Vollger MR}, Logsdon GA, Grothe R, et al. HiCanu: accurate assembly of segmental duplications, satellites, and allelic variants from high-fidelity long reads. Genome Res. 2020;30: 1291–1305. doi:10.1101/gr.263566.120}

\cvitem{\textbf{---}}{ Shafin K, Pesout T, Lorig-Roach R, Haukness M, Olsen HE, Bosworth C, et al. Nanopore sequencing and the Shasta toolkit enable efficient de novo assembly of eleven human genomes. Nat Biotechnol. 2020. doi:10.1038/s41587-020-0503-6 }

\cvitem{\textbf{---}}{Porubsky D, Ebert P, Audano PA, \textbf{Vollger MR}, Harvey WT, Munson KM, et al. A fully phased accurate assembly of an individual human genome. bioRxiv. 2019. p. 855049. doi:10.1101/855049}

\cvitem{\textbf{---}}{Sulovari A, Li R, Audano PA, Porubsky D, \textbf{Vollger MR}, Logsdon GA, et al. Human-specific tandem repeat expansion and differential gene expression during primate evolution. Proc Natl Acad Sci U S A. 2019; 201912175. doi:10.1073/pnas.1912175116}

\cvitem{\textbf{---}}{ Hsieh P, \textbf{Vollger MR}, Dang V, Porubsky D, Baker C, Cantsilieris S, Hoekzema K, Lewis AP, Munson KM, Sorensen M, Kronenberg ZN, Murali S, Nelson BJ, Chiatante G, Maggiolini FAM, Blanche H, Underwood JG, Antonacci F, Deleuze J-F, Eichler EE. 2019. Adaptive archaic introgression of copy number variants and the discovery of previously unknown human genes. Science 366. doi:10.1126/science.aax2083 }

\cvitem{\textbf{---}}{  Maggiolini FAM, Cantsilieris S, D Addabbo P, Manganelli M, Coe BP, Dumont BL, Sanders AD, Pang AWC, \textbf{Vollger MR}, Palumbo O, Palumbo P, Accadia M, Carella M, Eichler EE, Antonacci F. 2019. Genomic inversions and GOLGA core duplicons underlie disease instability at the 15q25 locus. PLoS Genet 15:e1008075. doi:10.1371/journal.pgen.1008075 }



%----------------------------------------------------------------------------------------
%	Invited Talks SECTION
%----------------------------------------------------------------------------------------

\section{\textbf{Invited Talks}}

\cventry{Sep 2020}{Plenary talk}{T2T and HPRC conference, National}{University of Washington}{}{A complete view of segmental duplications and their variation}


\cventry{Sep 2019}{Plenary talk}{Pacific Biosciences User Group Meeting, National}{University of Delaware}{}{Improved Assembly of Segmental Duplications Using HiFi}


\cventry{Feb 2015}{Speaker}{The Princeton High Throughput Sequencing Group, Local}{Princeton University}{}{Computational methods to quantify DNA damage done to Saccharomyces cerevisiae by UV and Cisplatin }



%----------------------------------------------------------------------------------------
%	Posters SECTION
%----------------------------------------------------------------------------------------

\section{\textbf{Posters}}

\cventry{December 2019}{Improved Assembly of Human Genomes Using HiFi}{Annual Scientific Meeting}{Howard Hughes Medical Institute}{}{Presented a poster on my research on assembly of human genomes using accurate long reads.}

\cventry{October 2018}{Resolving segmental duplications using long reads and correlation clustering}{Collaborative Seminar Series}{Allen Institute, Fred Hutch, and UW Medicine}{}{Presented a poster on my thesis research on developing and applying methods that use paralog specific variants (PSVs) to resolve collapsed duplications to improve genome assembly. }

\cventry{September 2017/2018}{Resolving Segmental Duplications with PSV based Community Detection}{Genome Sciences Annual Retreat}{Washington University}{}{Presented a poster on my thesis research on developing and applying methods that use paralog specific variants (PSVs) to resolve collapsed duplications to improve genome assembly.  }


\cventry{April 2017}{Identifying Multiple Charge States of Peptides in Mass Spectrometry}{2017 NHGRI Annual Meeting}{Washington University in St. Louis}{}{Presented a poster on the research I did with William Noble. A description of the research can be found in the Independent Work and Research section. }



%----------------------------------------------------------------------------------------
%	Other Experience
%----------------------------------------------------------------------------------------
%\section{\textbf{Other Employment}}
%None

%----------------------------------------------------------------------------------------
%	Research Experience
%----------------------------------------------------------------------------------------
\section{\textbf{Research Experience}}

\cventry{Spring 2018 - 2021}{Doctoral Candidate}{Resolving Duplications in Genome Assembly}{University of Washington}{}{Advisor Evan Eichler, Genome Sciences. Developing and applying methods that use paralog specific variants (PSVs) to resolve duplications and improve genome assembly. }

\cventry{Spring 2017 - Spring 2018}{Predoctoral Candidate}{See previous item}{University of Washington}{}{}

\cventry{Winter 2017}{Predoctoral Candidate}{ Identifying Insertion/Deletion Events in Mendelian Diseases}{University of Washington}{}{Advisor Debbie Nickerson, Genome Sciences. Implemented a variety of variant callers on a large number of genomes across many Mendelian Diseases in order to more consistently identify insertion and deletion events}

\cventry{Fall 2016}{Predoctoral Candidate}{Tandem Identification of Multiple Charge States in MS}{University of Washington}{}{Advisor William Noble, Genome Sciences.  Implemented group LASSO to confirm the existence of a single peptide in multiple charge states in mass spectrometry data for use in data independent acquisition (DIA) deconvolution. }

\cventry{Fall 2014 - Summer 2015}{Undergraduate Researcher}{Developing a Reference Genome for W303}{Princeton University}{}{Advisor Alison Gammie, Molecular Biology Department. Developed methods to create a reference genome for W303 \textit{Saccharomyces cerevisiae} using existing high-throughput sequencing data.}

\cventry{Fall 2014 - Summer 2015}{Undergraduate Researcher}{Quantifying Mutations Due to Cisplatin and UV}{Princeton University}{}{Advisor Alison Gammie, Molecular Biology Department. Developed computational methods to quantify DNA damage done to \textit{Saccharomyces cerevisiae} by UV and Cisplatin \textit{in vivo}.}

\cventry{Fall 2013}{Undergraduate Researcher}{Analysis of an Artificial Transcription Factor}{Princeton University}{}{Advisors Megan McClean, Alison Gammie, Marcus Noyes. Analyzed the transcriptomes of \textit{Saccharomyces cerevisiae} induced by Msn2 and by an artificial transcription factor mimicking Msn2.}



%----------------------------------------------------------------------------------------
%	COMPUTER SKILLS SECTION
%----------------------------------------------------------------------------------------

%\section{\textbf{Computer Skills}}

%\cvitem{Advanced}{Python, Snakemake, R}
%\cvitem{Intermediate}{C, C++, Java, \LaTeX}
%\cvitem{Familiar}{x86 Assembly, Objective C}

%----------------------------------------------------------------------------------------
%	References SECTION
%----------------------------------------------------------------------------------------

%\section{\textbf{References}}

%\cvitem{ \textcolor{blue}{\href{https://eichlerlab.gs.washington.edu/index.html}{E. Eichler}}}{eee@gs.washington.edu  }
%\cvitem{ \textcolor{blue}{\href{http://infotheory.ee.washington.edu/}{S. Kannan }}}{ksreeram@uw.edu    }
%\cvitem{ \textcolor{blue}{\href{http://chaissonlab.usc.edu/}{M. Chaisson}}}{mchaisso@usc.edu}

%\cvitem{ \textcolor{blue}{\href{http://molbio.princeton.edu/faculty/molbio-faculty/108-gammie}{A. Gammie}}}{agammie@princeton.edu  }
%\cvitem{ \textcolor{blue}{\href{https://www.cs.princeton.edu/~bwk/}{B. Kernighan }}}{bwk.cs@princeton.edu   }
%\cvitem{ \textcolor{blue}{\href{http://directory.engr.wisc.edu/bme/Faculty/Mcclean_Megan/}{M. McClean}}}{mmcclean@princeton.edu}





%----------------------------------------------------------------------------------------
%	COVER LETTER
%----------------------------------------------------------------------------------------

% To remove the cover letter, comment out this entire block

%\clearpage

%\recipient{HR Department}{Corporation\\123 Pleasant Lane\\12345 City, State} % Letter recipient
%\date{\today} % Letter date
%\opening{Dear Sir or Madam,} % Opening greeting
%\closing{Sincerely yours,} % Closing phrase
%\enclosure[Attached]{curriculum vit\ae{}} % List of enclosed documents

%\makelettertitle % Print letter title

%\lipsum[1-3] % Dummy text

%\makeletterclosing % Print letter signature

%----------------------------------------------------------------------------------------

\end{document}